%%%%%%%%%%%%%%%%%%%%%%%%%%%%%%%%%%%%%%%%%%%%%%%%%%%%%%%%%%
%% BEGIN PREAMBLE
\documentclass[10pt]{article}

%%%% Sets 1 inch margins on document
\usepackage[margin=1in]{geometry}

%%%% For math macros
\usepackage{amsmath}

%%%% Needed for including figures and other images
\usepackage{graphicx}

%%%% Adds ability to adjust document vertical spacing
% usage:
%   \setspace{1.5} % 1.5x for line spacing
\usepackage{setspace}

%%%% Needed for specifying the list items in enumerate env
% eg. (a,b,b) or (i,ii,iii), (1,2,3)
% usage:
%   \begin{enumerate} [label=(\alph*)] % for (a), (b), (c)
\usepackage{enumitem}

%%%% Defines Times New Roman as font
  % for math and text environments
%\usepackage{newtxtext,newtxmath}

  % actually use newcomputermodern
\usepackage{newcomputermodern}

%%%% For H float option when inserting figure
%   [H] inserts figure _exactly_ where it is typeset
% usage:
%   begin{figure} [H]
\usepackage{float}

%%%% For fancy header and footer ;)
\usepackage{fancyhdr}
\pagestyle{fancy}
\fancyhead[LO,L]{Samuel Barton}
\fancyhead[CO,C]{PHYS40 - Homework 7}
\fancyhead[RO,R]{\today}
\fancyfoot[LO,L]{}
\fancyfoot[CO,C]{\thepage}
\fancyfoot[RO,R]{}
\renewcommand{\headrulewidth}{0.4pt}
\renewcommand{\footrulewidth}{0.4pt}

%%%% Setting margins in tabular environments
% For making equations (esp. fractions) fit in cells vertically
\usepackage[column=C]{cellspace}
\cellspacetoplimit 4pt
\cellspacebottomlimit 4pt

%%%% define custom command for a raised chi 
\DeclareRobustCommand{\rchi}{{\mathpalette\irchi\relax}}
\newcommand{\irchi}[2]{\raisebox{\depth}{$#1\chi$}} % inner command, used by \rchi

%%%% For dirac notation
\usepackage{braket}

%%%% For sci. not. and units spacing
\usepackage{siunitx}
% define the angstrom as a unit
\DeclareSIUnit\angstrom{Å}

%% END PREAMBLE %%
%%%%%%%%%%%%%%%%%%%%%%%%%%%%%%%%%%%%%%%%%%%%%%%%%%%%%%%%%%%%%%%

\begin{document}

\setstretch{1.25} % set spacing to 1.25x

% Assignment Name
\begin{centering}
  \section*{HOMEWORK 7}
\end{centering}

\begin{enumerate}
\item 
  The average kinetic energy of a sample of gas molecules is a function of temperature and given by 
  \[
    KE = \frac{3 k_b T}{2}
  .\]
  Thus, we find that 
  \[
    T = \frac{2 \cdot KE}{3 k_b}
  .\]
  Plugging in values and setting $ KE = \qty{4.7}{\electronvolt} = \qty{7.52E-19}{\joule} $, we have
  
 \[
   T = \qty{3.63E4}{\kelvin}
 .\]

 At this temperature, the average kinetic energy of the molecules would exceed the binding energy, and the molecules would dissociate into free H atoms.
 
\item 
  The zero point energy is given by 
  \[
    E_{zp} = \frac{1}{2} h v_0
  ,\]
  where 
  \[
    v_0 = \frac{1}{2\pi} \sqrt{\frac{k}{\mu}}
  .\]
  Since $ \text{H}_2 $ and $ \text{HD} $ contain the same atoms, their $ k $ should be identical, meaning that only their reduced mass, $ \mu $, will be different.

  For $ \text{H}_2 $, we have 
  \[
    \mu_{H_2} = \frac{m_H \cdot m_H}{m_H + m_H} = \frac{m_H}{2}
  .\]
  For $ \text{HD} $, we have 
  \[
    \mu_{HD} = \frac{2m_H \cdot m_H}{2m_H + m_H} = \frac{2m_H}{3}
  .\]
  Thus, we have $ \mu_{HD} > \mu_{H_2} $, and therefore the zero point energy for $ \text{H}_2 $ is larger than that for $ \text{HD} $.
  
  The dissociation energy is equal to 
  \[
    E_d = V_0 - E_{zp}
  ,\]
  thus, the binding energy for $ \text{HD} $ is larger than the binding energy for $ \text{H}_2 $.

  Table 12-1 in E\&R confirms these results.
\item 
  \begin{enumerate}
  \item 
    We have the equation for inverse wavelength:
    \[
      \frac{1}{\lambda} = \frac{\hslash}{2\pi I c} l
    .\]
    Taking the reciprocal, we have
    \[
      \lambda = \frac{2\pi I c}{\hslash l}
    .\]
    Thus, for $ l=1 \to l = 2 $, we have 
    \[
      \lambda =\qty{1.47}{\milli\meter}
    .\]
    And for $ l=2 \to l = 3 $, we have 
    \[
      \lambda =\qty{0.983}{\milli\meter}
    \]
  \item 
    We have 
    \[
      r ^2 = \frac{h l}{\left( 2\pi \right)^2 \mu f} = \qty{1.21 E -47}{m ^2}
    ,\]
    and using the same internuclear distance, $ r $, we have the reduced mass for the different carbon isotope, 
    \[
      \mu = \frac{hl}{\left( 2\pi \right)^2 r ^2 f} = \qty{12.56}{amu}
    .\]
    Hence, the mass number for this isotope of carbon is $ 13 $.
  \item 
    We have the reduced mass, 
    \[
      \mu = \frac{m_C \cdot m_O}{m_C + m_O} = \qty{6.857}{amu} = \qty{1.14E -26}{\kg}
    .\]
    For the energy of ground state vibrations, we have 
    \[
      E = hf = \frac{1}{2} h v_0
    \]
    where
    \[
    v_0 = \frac{1}{2\pi} \sqrt{\frac{k}{\mu}}
    .\]
    Hence, we have 
    \[
      k = \left( 4 \pi f \right) ^2 \mu = \qty{7.41E3}{\kilogram\per\square\second}
    \]
    To find the amplitude, we have 
    \[
      \frac{1}{2} k A^2 = \frac{1}{2} h v_0
    ,\]
    thus, 
    \[
      A =  \sqrt{\frac{h}{2\pi k} \sqrt{\frac{k}{\mu}}} = \qty{3.39E-12}{\meter}
    .\]
    The carbon monoxide bond length is $ \qty{0.112}{\nano\meter} $, thus the amplitude of the ground-state vibrations is around \qty{3.03}{\percent} of the bond length.
  \end{enumerate}
\item 
  We start with the equation 
  \begin{align*}
    \frac{1}{\lambda} &= \frac{\hslash}{2\pi I c} l \\
     &= \frac{\hslash}{2 \pi c \mu r ^2} l
  \end{align*}
  Since we don't know the specific quantum numbers, $ l $, we must instead look at the transitions between these states. Hence, we use the equation 
  \[
    \frac{1}{\lambda_1} - \frac{1}{\lambda_2} = \frac{\hslash}{2 \pi c \mu r ^2} \cdot 1
  .\]
  Plugging in values for the wavelengths, we get the differences 
  \[
    \qtylist{1.28E-10;1.31E-10;1.29E-10;1.30E-10}{\meter}
  .\]
  Taking the average, we have 
  \[
    r \approx\qty{1.30E-10}{\meter}
  \]
  which is close to the actual value of \qty{1.27E-10}{\meter}.
\end{enumerate}

\end{document}

