%%%%%%%%%%%%%%%%%%%%%%%%%%%%%%%%%%%%%%%%%%%%%%%%%%%%%%%%%%
%% BEGIN PREAMBLE
\documentclass[10pt]{article}

%%%% Sets 1 inch margins on document
\usepackage[margin=1in]{geometry}

%%%% For math macros
\usepackage{amsmath}

%%%% Needed for including figures and other images
\usepackage{graphicx}

%%%% Adds ability to adjust document vertical spacing
% usage:
%   \setspace{1.5} % 1.5x for line spacing
\usepackage{setspace}

%%%% Needed for specifying the list items in enumerate env
% eg. (a,b,b) or (i,ii,iii), (1,2,3)
% usage:
%   \begin{enumerate} [label=(\alph*)] % for (a), (b), (c)
\usepackage{enumitem}

%%%% Defines Times New Roman as font
  % for math and text environments
%\usepackage{newtxtext,newtxmath}

  % actually use newcomputermodern
\usepackage{newcomputermodern}

%%%% For H float option when inserting figure
%   [H] inserts figure _exactly_ where it is typeset
% usage:
%   begin{figure} [H]
\usepackage{float}

%%%% For fancy header and footer ;)
\usepackage{fancyhdr}
\pagestyle{fancy}
\fancyhead[LO,L]{Samuel Barton}
\fancyhead[CO,C]{PHYS40 - Homework 2}
\fancyhead[RO,R]{\today}
\fancyfoot[LO,L]{}
\fancyfoot[CO,C]{\thepage}
\fancyfoot[RO,R]{}
\renewcommand{\headrulewidth}{0.4pt}
\renewcommand{\footrulewidth}{0.4pt}

%%%% Setting margins in tabular environments
% For making equations (esp. fractions) fit in cells vertically
\usepackage{cellspace}
\cellspacetoplimit 4pt
\cellspacebottomlimit 4pt

%%%% define custom command for a raised chi 
\DeclareRobustCommand{\rchi}{{\mathpalette\irchi\relax}}
\newcommand{\irchi}[2]{\raisebox{\depth}{$#1\chi$}} % inner command, used by \rchi

%%%% For dirac notation
\usepackage{braket}

%% END PREAMBLE %%
%%%%%%%%%%%%%%%%%%%%%%%%%%%%%%%%%%%%%%%%%%%%%%%%%%%%%%%%%%%%%%%

\begin{document}

\setstretch{1.25} % set spacing to 1.25x

% Assignment Name
\begin{centering}
  \section*{HOMEWORK 2}
\end{centering}

\begin{enumerate}
  \item For the Time-Independent Schrodinger Equation in 3-D, we have:
    \begin{equation*}
      - \frac{\hslash}{2 \mu} \nabla ^2 + U \left( x,y,z \right) \Psi \left( x,y,z \right) = E \Psi \left( x,y,z \right)
    \end{equation*}
    Plugging in the potential $ U \left( x,y,z \right) = \frac{1}{2} \mu \omega ^2 \left( x^2+y^2+z^2 \right) $, we have:
    \begin{equation*}
      - \frac{\hslash}{2 \mu} \nabla ^2 + \frac{1}{2} \mu \omega ^2 \left( x^2+y^2+z^2 \right)  \Psi \left( x,y,z \right) = E \Psi \left( x,y,z \right)
    \end{equation*}
    For using the separation of variables method, we must split the Schr{\"o}dinger equation into parts which depend only on $ x $, $ y $, and $ z $.

    First, we expand the "del" operator into its separate terms:
    \begin{equation*}
      - \frac{\hslash}{2 \mu} \left( \frac{\partial^2 }{\partial x^2}+\frac{\partial ^2}{\partial y^2} + \frac{\partial ^2}{\partial z^2} \right) + \frac{1}{2} \mu \omega ^2 \left( x^2+y^2+z^2 \right)  \Psi \left( x,y,z \right) = E \Psi \left( x,y,z \right)
    \end{equation*}
  \item 
  \item 
  \item 
  \item 
    \begin{align*}
      \hat{L}^2 Y_{1}^{-1}\left( \theta, \varphi \right) &= - \hslash^2 \left[ \frac{1}{\sin \theta} \frac{\partial }{\partial \theta} \left( \sin \theta \frac{\partial }{\partial \theta} \frac{1}{2} \sqrt{\frac{3}{2\pi}} \sin \theta e ^ {-i \varphi} \right) + \frac{1}{\sin ^2 \theta} \frac{\partial ^2}{\partial \varphi^2} \frac{1}{2}\sqrt{\frac{3}{2\pi}} \sin \theta e ^ {-i \varphi}\right] \\
      &=  - \hslash^2 \frac{1}{2} \sqrt{\frac{3}{2\pi}} \left[ \frac{1}{\sin \theta} \frac{\partial }{\partial \theta} \sin \theta \cos \theta e ^ {-i \varphi} - \frac{1}{\sin ^2 \theta} \sin \theta e ^ {-i \varphi}\right] \\
      &=  - \hslash^2 \frac{1}{2} \sqrt{\frac{3}{2\pi}} \left[ \frac{\cos^2\theta-\sin^2\theta}{\sin \theta} e ^ {-i \varphi} - \frac{1}{\sin \theta} e ^ {-i \varphi}\right] \\
      &=  - \hslash^2 \frac{1}{2} \sqrt{\frac{3}{2\pi}} e ^ {-i\varphi}\left[ \frac{1-2\sin^2\theta}{\sin \theta} - \frac{1}{\sin \theta} \right] \\
      &=  \hslash^2 \frac{1}{2} \sqrt{\frac{3}{2\pi}} e ^ {-i\varphi} 2 \sin\theta \\
      &= 2 \hslash^2 ~ Y_{1}^{-1} \left( \theta, \varphi \right)
    \end{align*}
    Thus, $ 2 \hslash ^2 $ is the eigenvalue.

    \begin{align*}
      \hat{L}_{z} Y_{1}^{-1}\left( \theta, \varphi \right) &= \frac{\hslash}{i} \frac{\partial }{\partial \varphi} \left[ \frac{1}{2} \sqrt{\frac{3}{2\pi}} \sin \theta e ^ {-i \varphi}\right] \\
      &= -i \frac{\hslash}{i} \frac{1}{2} \sqrt{\frac{3}{2\pi}} \sin \theta e ^ {-i \varphi} \\
      &= - \hslash ~ Y_{1}^{-1}\left( \theta, \varphi \right)
    \end{align*}
    Thus, $ -\hslash $ is the eigenvalue.

    \begin{align*}
      \hat{L}^2 Y_{2}^{2}\left( \theta, \varphi \right) &= - \hslash^2 \left[ \frac{1}{\sin \theta} \frac{\partial }{\partial \theta} \left( \sin \theta \frac{\partial }{\partial \theta} \frac{1}{4} \sqrt{\frac{15}{2\pi}} \sin ^2 \theta e ^ {2i \varphi} \right) + \frac{1}{\sin ^2 \theta} \frac{\partial ^2}{\partial \varphi^2} \frac{1}{4}\sqrt{\frac{15}{2\pi}} \sin ^2 \theta e ^ {2i \varphi}\right] \\
      &= - \hslash^2 \frac{1}{4} \sqrt{\frac{15}{2\pi}} e ^ {2i\varphi}\left[ \frac{1}{\sin \theta} \frac{\partial }{\partial \theta} 2 \sin^2\theta\cos\theta - \frac{4 \sin^2\theta}{\sin ^2 \theta} \right] \\
      &= - \hslash^2 \frac{1}{4} \sqrt{\frac{15}{2\pi}} e ^ {2i\varphi}\left[ \frac{1}{\sin \theta} \left( 4 \cos^2\theta \sin\theta - \sin^3\theta \right) - 4 \right] \\
      &= - \hslash^2 \frac{1}{4} \sqrt{\frac{15}{2\pi}} e ^ {2i\varphi}\left[ 4 \cos^2 \theta - \sin ^2 \theta - 4 \right] \\
      &= - \hslash^2 \frac{1}{4} \sqrt{\frac{15}{2\pi}} e ^ {2i\varphi}\left[ 4 - 4 \sin^2 \theta - \sin ^2 \theta - 4 \right] \\
      &= 5 \hslash^2 ~ Y_2^2 \left( \theta, \varphi \right)
    \end{align*}
    Thus, the eigenvalue is $ 5 \hslash ^2 $.

    \begin{align*}
      \hat{L}_{z} Y_{2}^{2}\left( \theta, \varphi \right) &= \frac{\hslash}{i} \frac{\partial }{\partial \varphi} \left[ \frac{1}{4} \sqrt{\frac{15}{2\pi}} \sin^2 \theta e ^ {2i \varphi}\right] \\
      &= 2\hslash ~ \hat{L}_{z} Y_{2}^{2}\left( \theta, \varphi \right) 
    \end{align*}
    Thus, the eigenvalue is $ 2 \hslash $.
\end{enumerate}

\end{document}

