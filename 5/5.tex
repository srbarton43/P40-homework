%%%%%%%%%%%%%%%%%%%%%%%%%%%%%%%%%%%%%%%%%%%%%%%%%%%%%%%%%%
%% BEGIN PREAMBLE
\documentclass[10pt]{article}

%%%% Sets 1 inch margins on document
\usepackage[margin=1in]{geometry}

%%%% For math macros
\usepackage{amsmath}

%%%% Needed for including figures and other images
\usepackage{graphicx}

%%%% Adds ability to adjust document vertical spacing
% usage:
%   \setspace{1.5} % 1.5x for line spacing
\usepackage{setspace}

%%%% Needed for specifying the list items in enumerate env
% eg. (a,b,b) or (i,ii,iii), (1,2,3)
% usage:
%   \begin{enumerate} [label=(\alph*)] % for (a), (b), (c)
\usepackage{enumitem}

%%%% Defines Times New Roman as font
  % for math and text environments
%\usepackage{newtxtext,newtxmath}

  % actually use newcomputermodern
\usepackage{newcomputermodern}

%%%% For H float option when inserting figure
%   [H] inserts figure _exactly_ where it is typeset
% usage:
%   begin{figure} [H]
\usepackage{float}

%%%% For fancy header and footer ;)
\usepackage{fancyhdr}
\pagestyle{fancy}
\fancyhead[LO,L]{Samuel Barton}
\fancyhead[CO,C]{PHYS40 - Homework 5}
\fancyhead[RO,R]{\today}
\fancyfoot[LO,L]{}
\fancyfoot[CO,C]{\thepage}
\fancyfoot[RO,R]{}
\renewcommand{\headrulewidth}{0.4pt}
\renewcommand{\footrulewidth}{0.4pt}

%%%% Setting margins in tabular environments
% For making equations (esp. fractions) fit in cells vertically
\usepackage[column=C]{cellspace}
\cellspacetoplimit 4pt
\cellspacebottomlimit 4pt

%%%% define custom command for a raised chi 
\DeclareRobustCommand{\rchi}{{\mathpalette\irchi\relax}}
\newcommand{\irchi}[2]{\raisebox{\depth}{$#1\chi$}} % inner command, used by \rchi

%%%% For dirac notation
\usepackage{braket}

\usepackage{siunitx}
\DeclareSIUnit\angstrom{Å}

%% END PREAMBLE %%
%%%%%%%%%%%%%%%%%%%%%%%%%%%%%%%%%%%%%%%%%%%%%%%%%%%%%%%%%%%%%%%

\begin{document}

\setstretch{1.25} % set spacing to 1.25x

% Assignment Name
\begin{centering}
  \section*{HOMEWORK 5}
\end{centering}

\begin{enumerate}
\item 
  From Example 9-2 we have the expanded equation:
 \begin{multline*}
   \psi_A = \frac{1}{\sqrt{3!}} [ \psi_ \alpha (1) \psi_ \beta(2) \psi_ \gamma(3) + \psi_ \beta(1) \psi_ \gamma(2) \psi_ \alpha(3) + \psi_ \gamma(1) \psi_ \alpha(2) \psi_ \beta(3) \\ 
   - \psi_ \gamma(1) \psi_ \beta (2) \psi_ \alpha(3) - \psi_ \beta(1) \psi_ \alpha (2) \psi_ \gamma (3) - \psi_ \alpha (1) \psi_ \gamma(2) \psi_ \beta(3) ]
 \end{multline*}
\begin{enumerate}
  \item Without loss of generality, let us swap particles 1 and 3. Now we have the equation:
  \begin{multline*}
    \psi_A' = \frac{1}{\sqrt{3!}} [ \psi_ \gamma (1) \psi_ \beta(2) \psi_ \alpha(3) + \psi_ \alpha(1) \psi_ \gamma(2) \psi_ \beta(3) + \psi_ \beta(1) \psi_ \alpha(2) \psi_ \gamma(3) \\ 
    - \psi_ \alpha(1) \psi_ \beta (2) \psi_ \gamma(3) - \psi_ \gamma(1) \psi_ \alpha (2) \psi_ \beta (3) - \psi_ \beta (1) \psi_ \gamma(2) \psi_ \alpha(3) ]
  \end{multline*}
  We see that: 
  \[
    \psi_A' = - \psi_A
  .\]
  Clearly, the same result is achieved by swapping particles 1 and 2, or particles 2 and 3.
  Thus, the expanded form of the wave function is antisymmetric with respect to an exchange of the labels of any two particles
  \item Without loss of generality, assume that particles 1 and 2 are in the same space and spin quantum state.
    Now er have the equation:
  \begin{multline*}
    \psi_A' = \frac{1}{\sqrt{3!}} [ \psi_ \alpha (1) \psi_ \alpha(2) \psi_ \gamma(3) + \psi_ \alpha(1) \psi_ \gamma(2) \psi_ \alpha(3) + \psi_ \gamma(1) \psi_ \alpha(2) \psi_ \alpha(3) \\ 
    - \psi_ \gamma(1) \psi_ \alpha (2) \psi_ \alpha(3) - \psi_ \alpha(1) \psi_ \alpha (2) \psi_ \gamma (3) - \psi_ \alpha (1) \psi_ \gamma(2) \psi_ \alpha(3) ]
  \end{multline*}
  Here we see that all terms cancel, and that 
  \[
    \psi_A' = 0
  .\]
  The same result occurs if any other particles are in the same space and spin quantum state.
\end{enumerate}
\item
  For the particle in a box, we have ground energy state, 
  \[
    E_0 = \frac{h^2}{8ma^2}
  .\]
  For $ a=\qty{2}{\angstrom} $, we have,
  \[
    E_0=\qty{9.4}{\electronvolt}
  .\]
  For quantum numbers $ n_x,n_y,n_z $, we have,
  \[
    E_{n_xn_yn_z} = \left( n_x^2+n_y^2+n_z^2 \right) E_0 
  .\]
  \begin{enumerate}
  \item
    If the particles are fermions, they must obey Pauli's Exclusion Principle.
    Therefore, two particles can be in each state with one having ``spin up", and the other having ``spin down".
    Thus, for the lowest energy of the system we have the states,
    \[
      (1,1,1), (1,1,2), (1,2,1), (2,1,1)
    .\]
    These quantum numbers yield the energies,
    \begin{gather*}
      E_{111} = \left( \qty{9.4}{\electronvolt} \right) \left(1^2 + 1^2 + 1^2\right) = \qty{28.2}{\electronvolt} \\
      E_{112}=E_{121}=E_{211} = \left( \qty{9.4}{\electronvolt} \right) \left( 2^2+1^2+1^2 \right) = \qty{56.4}{\electronvolt}
    .\end{gather*}
    And thus,
    \[
      E=2 \left( \qty{28.2}{\electronvolt} + 3\times \qty{56.4}{\electronvolt} \right) = \qty{398.4}{\electronvolt}
    .\]
    \item
      Bosons do not follow the Pauli Exclusion Principle, so all 8 particles can occupy the ground state,
      \[
        (1,1,1)
      .\]
      Therefore,
      \[
        E=8\times E_{111} = \qty{225.6}{\electronvolt}
      .\]
  \end{enumerate}
\item
  \begin{enumerate}
  \item 
    \begin{align*}
      \mathcal{P} \left( x_1,x_2 \right) &= \left| \psi \left( x_1,x_2 \right) \right|^2 \\
                                     &= \frac{1}{2} \left[ \psi_n \left( x_1 \right) \psi_m \left( x_2 \right) \pm \psi_n \left( x_2 \right) \psi_m \left( x_1 \right) \right]^2 \\
                                     &= \frac{1}{2} \left[ \left|\psi_n \left( x_1 \right) \right| ^2 \left| \psi_m \left( x_2 \right) \right| ^2 + \left| \psi_n \left( x_2 \right) \right| ^2 \left| \psi_m \left( x_1 \right) \right| ^2 \pm 2 \psi_n \left( x_1 \right) \psi_m \left( x_2 \right) \psi_n \left( x_2 \right) \psi_m \left( x_1 \right) \right]
    \end{align*}
  \item
    \begin{align*}
      P_{left} ={}& \int_{0}^{a / 2} \int_{0}^{a / 2} \mathcal{P} \left( x_1,x_2 \right) d x_1 d x_2 \\
      \begin{split} 
        ={}& \int_{0}^{a / 2} \int_{0}^{a / 2} \frac{1}{2} \bigl[ \left|\psi_n \left( x_1 \right) \right| ^2 \left| \psi_m \left( x_2 \right) \right| ^2 + \left| \psi_n \left( x_2 \right) \right| ^2 \left| \psi_m \left( x_1 \right) \right| ^2 \\
        & + 2 \psi_n \left( x_1 \right) \psi_m \left( x_2 \right) \psi_n \left( x_2 \right) \psi_m \left( x_1 \right) \bigr] d x_1 d x_2
      \end{split}\\
      \begin{split} 
        ={}& \frac{1}{2} \int_{0}^{a / 2} \left|\psi_n \left( x_1 \right) \right| ^2 d x_1 \int_{0}^{a / 2} \left| \psi_m \left( x_2 \right) \right| ^2 d x_2 + \frac{1}{2} \int_{0}^{a / 2} \left| \psi_n \left( x_2 \right) \right| ^2 d x_2 \int_{0}^{a / 2} \left| \psi_m \left( x_1 \right) \right| ^2 d x_1 \\
        & + \int_{0}^{a / 2} \psi_n \left( x_1 \right) \psi_m \left( x_1 \right)  d x_1 \int_{0}^{a / 2} \psi_m \left( x_2 \right) \psi_n \left( x_2 \right)  d x_2
      \end{split} \\
      \begin{split} 
        ={}& \frac{2}{a^2} \bigl[ \int_{0}^{a / 2} \cos^2 \left( \frac{\pi x_1}{a} \right) d x_1 \int_{0}^{a / 2} \sin^2 \left( \frac{\pi x_2}{a} \right) d x_2 +  \int_{0}^{a / 2} \sin ^2 \left( \frac{\pi x_1}{a} \right) d x_2 \int_{0}^{a / 2} \cos ^2 \left( \frac{\pi x_1}{a} \right) d x_1 \\
           & + 2 \int_{0}^{a / 2} \sin \left( \frac{\pi x_1}{a} \right) \cos \left( \frac{\pi x_1}{a} \right) d x_1 \int_{0}^{a / 2} \cos \left( \frac{\pi x_2}{a} \right) \sin \left( \frac{\pi x_2}{a} \right)  d x_2 \bigr]
      \end{split} \\
                  &= \frac{2}{a^2} \left[ \frac{a^2}{16} + \frac{a^2}{16} + \frac{a^2}{2\pi^2} \right] \\
                  &= \frac{1}{4} + \frac{1}{\pi^2} \approx \num{0,351}
    \end{align*}
  \item 
    For the antisymmetric case, we see that we simply must subtract the rightmost term instead of adding it.
    Thus, we have
    \[
     P_{left} = \frac{1}{4} - \frac{1}{\pi^2} \approx \num{0,149}
    .\]
  \end{enumerate}
\item 
  \begin{enumerate}
  \item 
    We know that the integral of an odd function over all space is equal to zero, whereas the integral of an even function over all space is nonzero.
    Here we notice that we have double integral with each integral consisting of a polynomial multiplied by an exponential of the form $ e ^ {-x^2} $.
    Since the exponential is an even function, the overall integral will be odd if either of the polynomial functions are odd functions, i.e., they are to an odd power.
    Therefore, the integral of that form will be zero unless both $ n $ and $ m $ are even.
  \item 
    
    Both integrals fits the form of the one from lecture 9, 
    \[
      \int_{-\infty}^{\infty} u^{n} e ^ {-u^2}du
    ,\]
    and we know the solution to this integral for even $ n $ is 
    \[
      \int_{-\infty}^{\infty} u^{n} e ^ {-u^2}du = \left( - \frac{d}{da} \right) ^{n / 2} \sqrt{\frac{\pi}{a}} \Bigg|_{a=1}
    \]
    Thus,
    \begin{equation*}
      \int_{-\infty}^{\infty} \zeta ^2 e ^ {-\zeta^2}d \zeta = \frac{\sqrt{\pi}}{2}
    ,\end{equation*}
    And,
    \begin{equation*}
      \int_{-\infty}^{\infty} \zeta ^4 e ^ {-\zeta^2}d \zeta = \frac{3\sqrt{\pi}}{4}
    .\end{equation*}
  \item 
    \begin{align*}
      I &= \int_{-\infty}^{\infty} \int_{-\infty}^{\infty} \left| \psi_n \left( x_1 \right) \right|^2 \hat{W} \left| \psi_n \left( x_1 \right) \right| ^2 d x_1 d x_2  \\
      &= -\frac{1}{2} m \omega_0 ^2 \int_{-\infty}^{\infty} \int_{-\infty}^{\infty} \frac{1}{4\pi x_0^2} \zeta_2^2 e ^ {- \left( \zeta_1^2+\zeta_2^2 \right)} \left( x_1-x_2 \right) ^2 d x_1 d x_2 \\
      &= -\frac{1}{2} m \omega_0 ^2 \int_{-\infty}^{\infty} \int_{-\infty}^{\infty} \frac{1}{4\pi x_0^2} \zeta_2^2 e ^ {- \left( \zeta_1^2+\zeta_2^2 \right)} \left( x_1^2-2x_1x_2+x_2^2 \right) d x_1 d x_2 \\
      &= -\frac{1}{8\pi} m \omega_0 ^2 x_0^2 \int_{-\infty}^{\infty} \int_{-\infty}^{\infty}  \zeta_2^2 e ^ {- \left( \zeta_1^2+\zeta_2^2 \right)} \left( \zeta_1^2-2\zeta_1\zeta_2+\zeta_2^2 \right) d \zeta_1 d \zeta_2 \\
      &= -\frac{1}{8\pi} m \omega_0 ^2 x_0 ^2 \int_{-\infty}^{\infty} \int_{-\infty}^{\infty}  \zeta_1^2 \zeta_2^2 e ^ {- \left( \zeta_1^2+\zeta_2^2 \right)}-2\zeta_1\zeta_2 e ^ {- \left( \zeta_1^2+\zeta_2^2 \right)}+\zeta_2^2e ^ {- \left( \zeta_1^2 +\zeta_2^2 \right)}  d \zeta_1 d \zeta_2 \\
      &= -\frac{1}{8\pi} m \omega_0 ^2 x_0^2 \left[ \frac{\pi}{4} - 0 +  \frac{3\pi x_0}{4} \right] \\
      &= -\frac{1}{32} m \omega_0 ^2 x_0^2 \left( 1 + 3 x_0 \right)
    \end{align*}

    \begin{align*}
      J &= \int_{-\infty}^{\infty} \int_{-\infty}^{\infty} \psi_n \left( x_1 \right) ^* \psi_m \left( x_2 \right) \hat{W} \psi_n \left( x_2 \right) ^* \psi_m \left( x_1 \right) d x_1 d x_2 \\
      &= -\frac{1}{8\pi x_0 ^2} m \omega_0 ^2 \int_{-\infty}^{\infty} \int_{-\infty}^{\infty} \zeta_1 \zeta_2 e ^ {- \left( \zeta_1^2+\zeta_2^2 \right)} \left( x_1-x_2 \right)^2 d x_1 d x_2 \\
      &= -\frac{1}{8\pi} m \omega_0 ^2 x_0^2 \int_{-\infty}^{\infty} \int_{-\infty}^{\infty} \zeta_1 \zeta_2 e ^ {- \left( \zeta_1^2+\zeta_2^2 \right)} \left( \zeta_1^2 - 2 \zeta_1 \zeta_2 + \zeta_2 ^2 \right) d x_1 d x_2 \\
      &= -\frac{1}{8\pi} m \omega_0 ^2 x_0^2 \left[ 0 - \frac{\pi}{2} + 0 \right] \\
      &= \frac{1}{16\pi} m \omega_0 ^2 x_0^2 
    \end{align*}

    Since for the symmetric spatial state, 
    \[
      \Delta E = I + J
    \]
    and for the antisymmetric spatial state,
    \[
      \Delta E = I - J
    ,\]
    we see that particles in a symmetric spatial state have a higher energy than ones in an antisymmetric state.
    
  \end{enumerate}
\end{enumerate}
\end{document}
