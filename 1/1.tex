%%%%%%%%%%%%%%%%%%%%%%%%%%%%%%%%%%%%%%%%%%%%%%%%%%%%%%%%%%
%% BEGIN PREAMBLE
\documentclass[10pt]{article}

%%%% Sets 1 inch margins on document
\usepackage[margin=1in]{geometry}

%%%% For math macros
\usepackage{amsmath}

%%%% Needed for including figures and other images
\usepackage{graphicx}

%%%% Adds ability to adjust document vertical spacing
% usage:
%   \setspace{1.5} % 1.5x for line spacing
\usepackage{setspace}

%%%% Needed for specifying the list items in enumerate env
% eg. (a,b,b) or (i,ii,iii), (1,2,3)
% usage:
%   \begin{enumerate} [label=(\alph*)] % for (a), (b), (c)
\usepackage{enumitem}

%%%% Defines Times New Roman as font
  % for math and text environments
\usepackage{newtxtext,newtxmath}

%%%% For H float option when inserting figure
%   [H] inserts figure _exactly_ where it is typeset
% usage:
%   begin{figure} [H]
\usepackage{float}

%%%% For fancy header and footer ;)
\usepackage{fancyhdr}
\pagestyle{fancy}
\fancyhead[LO,L]{Samuel Barton}
\fancyhead[CO,C]{PHYS40 - Homework 1}
\fancyhead[RO,R]{\today}
\fancyfoot[LO,L]{}
\fancyfoot[CO,C]{\thepage}
\fancyfoot[RO,R]{}
\renewcommand{\headrulewidth}{0.4pt}
\renewcommand{\footrulewidth}{0.4pt}

%%%% Setting margins in tabular environments
% For making equations (esp. fractions) fit in cells vertically
\usepackage{cellspace}
\cellspacetoplimit 4pt
\cellspacebottomlimit 4pt
%% END PREAMBLE %%
%%%%%%%%%%%%%%%%%%%%%%%%%%%%%%%%%%%%%%%%%%%%%%%%%%%%%%%%%%%%%%%

\begin{document}

\setstretch{1.25} % set spacing to 1.25x

% Assignment Name
\begin{centering}
  \section*{HOMEWORK 1}
\end{centering}

\begin{enumerate}
  \item 
    \begin{enumerate}
      \item 
      \begin{align*}
        &\frac{1}{2} \left(  e ^ {i \theta } + e ^ {- i \theta } \right) \\
        &= \frac{1}{2} \left( \cos \theta + i \sin \theta + \frac{1}{\cos \theta + i \sin \theta} \right) \\
        &= \frac{1}{2} \left( \frac{\cos ^2 \theta + i \sin \theta \cos \theta}{\cos \theta + i \sin \theta } + \frac{i \sin \theta \cos \theta - \sin ^2 \theta }{\cos \theta i \sin \theta } + \frac{1}{\cos \theta + i \sin \theta }   \right) \\
        &= \frac{1}{2} \left( \frac{2 \cos ^2 \theta + 2i \cos \theta \sin \theta }{\cos \theta + i \sin \theta } \right) \\
        &= \frac{2 \cos \theta }{2} \left( \frac{\cos \theta +i \sin \theta }{\cos \theta +i \sin \theta } \right) \\
        &= \cos  \theta 
      \end{align*}
      \begin{align*}
        &\frac{1}{2i} \left( e ^ {i \theta } - e ^ {-i \theta } \right) \\
        &= \frac{1}{2i} \left( \cos \theta +i\sin \theta - \frac{1}{\cos \theta +i\sin \theta } \right) \\
        &= \frac{1}{2} \left( \frac{\cos ^2 \theta + i \sin \theta \cos \theta}{\cos \theta + i \sin \theta } + \frac{i \sin \theta \cos \theta - \sin ^2 \theta }{\cos \theta i \sin \theta } - \frac{1}{\cos \theta + i \sin \theta }   \right) \\
        &= \frac{1}{2i} \left( \frac{-2\sin  ^2 \theta +2i\sin \theta \cos \theta }{\cos \theta +i\sin \theta } \right) \\
        &= \sin \theta \left( \frac{-\sin \theta + i\cos \theta }{-\sin \theta + i\cos \theta } \right) \\
        &= \sin \theta 
      \end{align*}
    \item 
    \end{enumerate}
\end{enumerate}

\end{document}

