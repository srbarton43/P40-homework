%%%%%%%%%%%%%%%%%%%%%%%%%%%%%%%%%%%%%%%%%%%%%%%%%%%%%%%%%%
%% BEGIN PREAMBLE
\documentclass[10pt]{article}

%%%% Sets 1 inch margins on document
\usepackage[margin=1in]{geometry}

%%%% For math macros
\usepackage{amsmath}

%%%% Needed for including figures and other images
\usepackage{graphicx}

%%%% Adds ability to adjust document vertical spacing
% usage:
%   \setspace{1.5} % 1.5x for line spacing
\usepackage{setspace}

%%%% Needed for specifying the list items in enumerate env
% eg. (a,b,b) or (i,ii,iii), (1,2,3)
% usage:
%   \begin{enumerate} [label=(\alph*)] % for (a), (b), (c)
\usepackage{enumitem}

%%%% Defines Times New Roman as font
  % for math and text environments
\usepackage{newtxtext,newtxmath}

%%%% For H float option when inserting figure
%   [H] inserts figure _exactly_ where it is typeset
% usage:
%   begin{figure} [H]
\usepackage{float}

%%%% For fancy header and footer ;)
\usepackage{fancyhdr}
\pagestyle{fancy}
\fancyhead[LO,L]{Samuel Barton}
\fancyhead[CO,C]{PHYS40 - Homework 1}
\fancyhead[RO,R]{\today}
\fancyfoot[LO,L]{}
\fancyfoot[CO,C]{\thepage}
\fancyfoot[RO,R]{}
\renewcommand{\headrulewidth}{0.4pt}
\renewcommand{\footrulewidth}{0.4pt}

%%%% Setting margins in tabular environments
% For making equations (esp. fractions) fit in cells vertically
\usepackage{cellspace}
\cellspacetoplimit 4pt
\cellspacebottomlimit 4pt

%%%% define custom command for a raised chi 
\DeclareRobustCommand{\rchi}{{\mathpalette\irchi\relax}}
\newcommand{\irchi}[2]{\raisebox{\depth}{$#1\chi$}} % inner command, used by \rchi

%% END PREAMBLE %%
%%%%%%%%%%%%%%%%%%%%%%%%%%%%%%%%%%%%%%%%%%%%%%%%%%%%%%%%%%%%%%%

\begin{document}

\setstretch{1.25} % set spacing to 1.25x

% Assignment Name
\begin{centering}
  \section*{HOMEWORK 1}
\end{centering}

\begin{enumerate}
  \item 
    \begin{enumerate}
      \item 
      \begin{align*}
        &\frac{1}{2} \left(  e ^ {i \theta } + e ^ {- i \theta } \right) \\
        =& \frac{1}{2} \left( \cos \theta + i \sin \theta + \frac{1}{\cos \theta + i \sin \theta} \right) \\
        =& \frac{1}{2} \left( \frac{\cos ^2 \theta + i \sin \theta \cos \theta}{\cos \theta + i \sin \theta } + \frac{i \sin \theta \cos \theta - \sin ^2 \theta }{\cos \theta i \sin \theta } + \frac{1}{\cos \theta + i \sin \theta }   \right) \\
        =& \frac{1}{2} \left( \frac{2 \cos ^2 \theta + 2i \cos \theta \sin \theta }{\cos \theta + i \sin \theta } \right) \\
        =& \frac{2 \cos \theta }{2} \left( \frac{\cos \theta +i \sin \theta }{\cos \theta +i \sin \theta } \right) \\
        =& \cos  \theta 
      \end{align*}
      \begin{align*}
        &\frac{1}{2i} \left( e ^ {i \theta } - e ^ {-i \theta } \right) \\
        =& \frac{1}{2i} \left( \cos \theta +i\sin \theta - \frac{1}{\cos \theta +i\sin \theta } \right) \\
        =& \frac{1}{2} \left( \frac{\cos ^2 \theta + i \sin \theta \cos \theta}{\cos \theta + i \sin \theta } + \frac{i \sin \theta \cos \theta - \sin ^2 \theta }{\cos \theta i \sin \theta } - \frac{1}{\cos \theta + i \sin \theta }   \right) \\
        =& \frac{1}{2i} \left( \frac{-2\sin  ^2 \theta +2i\sin \theta \cos \theta }{\cos \theta +i\sin \theta } \right) \\
        =& \sin \theta \left( \frac{-\sin \theta + i\cos \theta }{-\sin \theta + i\cos \theta } \right) \\
        =& \sin \theta 
      \end{align*}
    \item 
      The probability density, $ P \left( x \right) $ is defined:
      \begin{equation*}
        P \left( x \right) = \psi ^* \left( x \right)\psi \left( x \right)
      \end{equation*}
      So we have:
      \begin{align*}
        P \left( x \right) &= \rchi (x) e ^ {-i \varphi } \rchi (x) e ^ {i \varphi } \\
                           &= \left[ \rchi (x)  \right] ^2
      \end{align*}
      Hence we see that the probability density is independent of the global complex phase factor.
    \item 
      \begin{align*}
        & \sin \theta \sin \varphi \\
        = & \frac{1}{2i}\frac{1}{2i}\left( e ^ {i\theta }-e ^ {-i\theta } \right) \left( e ^ {i\varphi }-e ^ {-i\varphi } \right) \\
        =& -\frac{1}{4} \left[ e ^ {i\theta +i\varphi } - e ^ {i\theta -i\varphi } - e ^ {i\varphi -i\theta } + e ^ {-i\theta -i\varphi } \right] \\
        =& -\frac{1}{4} \left[ 2\cos \left( \theta +\varphi  \right) - 2\cos \left( \theta -\varphi  \right) \right] \\
        =& \frac{1}{2} \left[ \cos \left( \theta -\varphi  \right) - \cos \left( \theta +\varphi  \right) \right]
      \end{align*}
    \end{enumerate}
  \item
  \begin{enumerate}
    
    \item 
    To show that the eigenstates of the infinite square well have been normalized between $ x=0 $ and $ x=a $, we must show that its inner product is one over that region:

    \begin{align*}
      & \int_{0}^{a} \left| \psi_n \right|^2 dx \\
      =& \frac{2}{a} \int_{0}^{a} \sin ^2 \left( \frac{n \pi x}{a} \right) dx\\
      =& \frac{2}{a} \left[ \frac{x}{2} - \frac{a \sin \left( \frac{2 n\pi x}{a} \right)}{4 n\pi} \right]_0^a \\
      =& \frac{2}{a} \cdot \frac{a}{2} \\
      =& 1
    \end{align*}
    Hence the eigenstates are normalized.
  \item 
    \begin{align*}
      &\int_{0}^{a} \psi_n \psi_m dx \\
     =&~\frac{2}{a} \int_{0}^{a} \sin \left( \frac{n \pi x}{a} \right) \sin \left( \frac{m \pi x}{a} \right) dx \\ 
     =&~a \int_{0}^{a} \left[ \cos \left( \frac{\left( n-m \right)\pi}{a}x \right) - \cos \left( \frac{\left( n+m \right)\pi}{a}x \right) \right]dx && \text{(Using the trig identity from 1c)} \\
     =&~a \left[ \frac{a}{\left( n-m \right)\pi} \sin \left( \frac{\left( n-m \right)\pi}{a}x \right) - \frac{a}{\left( n+m \right)\pi} \sin \left( \frac{\left( n+m \right)\pi}{a}x \right) \right]_0^a \\
     =&~0 && \text{(Because m and n are integers)}
   \end{align*}
  \item 
    To show that the eigenstates for the QHM are orthogonal, $ \psi_0 $ and $ \psi_2 $, we must show that the expression $ \int_{-\infty}^{\infty} \psi_0 \psi_2 dx = 0$. \\
    First substitute $ H_0 \left( \xi \right)=1 $ and $ H_2 \left( \xi \right)=4\xi^2-2 $ and put everything in terms of $ \xi $.
    We also drop all normalization constants for clarity since they do not matter when showing orthogonality:
    \begin{align*}
      &\psi_0 \left( x \right) = e ^ {\frac{1}{2}\xi ^2} && \psi_2 \left( x \right) = \left( 4 \xi - 2 \right) e ^ {- \frac{1}{2} \xi}
    \end{align*}

    \begin{align*}
      &~\int_{-\infty}^{\infty} \psi_0 \psi_2 dx \\
     =&~\int_{-\infty}^{\infty} e ^ {\frac{1}{2}\xi ^2} \left( 4 \xi - 2 \right) e ^ {- \frac{1}{2} \xi} d\xi \\
     =&~ \int_{-\infty}^{\infty} \left[ 4\xi^2 e ^ {-\xi^2} -2e ^ {-\xi^2} \right]d\xi \\
     =&~ \left[ -2\xi e ^ {-\xi^2} d\xi \right]_{-\infty}^{\infty} \\
     =&~ 0
    \end{align*}
    Hence, we have shown that the two eigenstates are orthogonal.
    
  \end{enumerate}
  \item 
    \begin{enumerate}
      \item 
    \end{enumerate}
  \item 
  \begin{itemize}
    \item
      \textit{(6-30)} \\
      The zero-point energy is given by:
      \begin{align*}
        E_0&=\frac{1}{2} \hslash \omega \\
           &= \frac{1}{2}\hslash \sqrt{\frac{C}{m}} \\
           &= \frac{1}{2} \left( 1.05 \times 10 ^ {-34} \right) \sqrt{\frac{10^3}{4.1 \times 10^{-26}}} && \text{Plugging in numbers} \\
           &\approx 8.20 \times 10 ^{-21}~\text{J} \quad \text{or} \quad 0.051~\text{eV}
      \end{align*}
    \item 
      \textit{(6-31)}
      \begin{enumerate}
      \item To get the discrete energy levels for the SHO, we have the equation:
        \begin{equation*}
          E_n=\left( n+\frac{1}{2} \right)\omega
        \end{equation*}
        Thus,
        \begin{equation*}
          E_{photon} = \Delta E = E_1-E_0 \approx 0.154~\text{eV} - 0.051~\text{eV} = 0.103~\text{eV}
        \end{equation*}
      \item We expect that the energy emitted is equal to the difference in energy between the ground state and first energy level, or $ 0.103~\text{eV} $.
      \item 
        For the photon,
        \begin{equation*}
          E_{photon} = \hslash \omega_{photon}
        \end{equation*}
        And, we also know that
        \begin{align*}
          &E_{photon} = \Delta E = \hslash \omega && \text{(where $ \omega $ is the classical oscillation frequency)}
        \end{align*}
        Thus,
        \begin{equation*}
          \omega_{photon} = \omega
        \end{equation*}
        We get frequency, $ f $:
        \begin{equation*}
          f = \frac{E_{photon}}{h} = \frac{0.102 \cdot 1.6 \times 10^{-19}}{6.626 \times 10^{-34}} = 2.46 \times 10^{13}~\text{Hz}
        \end{equation*}
        This frequency ($ \lambda \approx 12000~\text{nm} $) corresponds with infrared light.
      \end{enumerate}
  \end{itemize}
\end{enumerate}

\end{document}

