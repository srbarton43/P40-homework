%%%%%%%%%%%%%%%%%%%%%%%%%%%%%%%%%%%%%%%%%%%%%%%%%%%%%%%%%%
%% BEGIN PREAMBLE
\documentclass[10pt]{article}

%%%% Sets 1 inch margins on document
\usepackage[margin=1in]{geometry}

%%%% For math macros
\usepackage{amsmath}

%%%% Needed for including figures and other images
\usepackage{graphicx}

%%%% Adds ability to adjust document vertical spacing
% usage:
%   \setspace{1.5} % 1.5x for line spacing
\usepackage{setspace}

%%%% Needed for specifying the list items in enumerate env
% eg. (a,b,b) or (i,ii,iii), (1,2,3)
% usage:
%   \begin{enumerate} [label=(\alph*)] % for (a), (b), (c)
\usepackage{enumitem}

%%%% Defines Times New Roman as font
  % for math and text environments
%\usepackage{newtxtext,newtxmath}

  % actually use newcomputermodern
\usepackage{newcomputermodern}

%%%% For H float option when inserting figure
%   [H] inserts figure _exactly_ where it is typeset
% usage:
%   begin{figure} [H]
\usepackage{float}

%%%% For fancy header and footer ;)
\usepackage{fancyhdr}
\pagestyle{fancy}
\fancyhead[LO,L]{Samuel Barton}
\fancyhead[CO,C]{PHYS40 - Homework 4}
\fancyhead[RO,R]{\today}
\fancyfoot[LO,L]{}
\fancyfoot[CO,C]{\thepage}
\fancyfoot[RO,R]{}
\renewcommand{\headrulewidth}{0.4pt}
\renewcommand{\footrulewidth}{0.4pt}

%%%% Setting margins in tabular environments
% For making equations (esp. fractions) fit in cells vertically
\usepackage{cellspace}
\cellspacetoplimit 4pt
\cellspacebottomlimit 4pt

%%%% define custom command for a raised chi 
\DeclareRobustCommand{\rchi}{{\mathpalette\irchi\relax}}
\newcommand{\irchi}[2]{\raisebox{\depth}{$#1\chi$}} % inner command, used by \rchi

%%%% For dirac notation
\usepackage{braket}

%% END PREAMBLE %%
%%%%%%%%%%%%%%%%%%%%%%%%%%%%%%%%%%%%%%%%%%%%%%%%%%%%%%%%%%%%%%%

\begin{document}

\setstretch{1.25} % set spacing to 1.25x

% Assignment Name
\begin{centering}
  \section*{HOMEWORK 4}
\end{centering}

\begin{enumerate}
\item 
  \begin{enumerate}
  \item 
    With $ E = \mu \cdot B $ and considering the spin up and spin down cases,
    we have:
    \begin{align*}
      \Delta E &= \mu_B B \cos(0) - \mu_B B \cos(\pi) \\
               &= 2 \mu_B B \\
               &= 2 \frac{e \hslash}{2m_e} B \\
               &= 2.24 \times 10^{-23}~\text{J} = 1.40 \times 10^{-4}~\text{eV}
    \end{align*}
  \item 
    The energy required for the electron to flip spin is equal to the $ \Delta E $ found in the last question.
    Using the equation $ E=\frac{hc}{\lambda} $, we have:
    \begin{align*}
      \lambda &= \frac{hc}{\Delta E} \\
              &= 8.94 \times 10^{-3}~\text{m}
    \end{align*}
    The wavelength which causes the electrons' spin to flip is $ 8.94~\text{mm} $
  \end{enumerate}
\item
  We have:
  \begin{equation*}
    \Delta E = 2 \mu_B B = \frac{hc}{\lambda}
  \end{equation*}
  Thus,
  \begin{equation*}
    B = \frac{hc}{\lambda} \frac{1}{2 \mu_B} = .0511~\text{T}
  \end{equation*}
\item 
  For the states with $ \ell=4 $ and $ s=1/2 $, The state with the largest possible $ j $ and largest possible $ m_j $ has $ j=9 / 2 $ and $ m_j = 9 / 2 $.
  \begin{enumerate}
  \item   
  We have:
  \begin{gather*}
    \vec{J} = \vec{L} + \vec{S} \\
    |\vec{J}| = \sqrt{j(j+1)}\hslash = \sqrt{99} \hslash / 2 \\
    |\vec{L}| = \sqrt{\ell(\ell+1)}\hslash = \sqrt{120} \hslash\\
    |\vec{S}| = \sqrt{s(s+1)}\hslash = \sqrt{3} \hslash / 2
  \end{gather*}
  Now, apply the law of cosines
  \begin{align*}
    J^2 &= L^2+S^2 - 2 L S \cos (\pi - \theta) \\
    \frac{99}{4} &= 20+\frac{3}{4} + 2 \sqrt{15} \cos \theta \\
    \theta &= \cos ^{-1} \left( \frac{2}{\sqrt{15}} \right) = 1.03~\text{rad}\text{ or }58.9^\circ
  \end{align*}
  \item 
    $ \vec{\mu_{\ell}} $ is antiparallel to $ \vec{L} $ and $ \vec{\mu_{s}} $ is antiparallel to $ \vec{S} $, thus the angle is the same as in part (a), $ \theta = 58.9^\circ $.
  \item 
    \begin{equation*}
      \cos \theta = \frac{J_z}{|\vec{J}|} = \frac{9}{\sqrt{99}}
    \end{equation*}
    Thus, $ \theta = 25.24^\circ $.
  \end{enumerate}
\item 
  \begin{enumerate}
  \item 
    For the \textit{3d} level, we know that $ n=3 $, $ \ell = 2 $, and $ s=1 / 2 $.
    
    Therefore, $ j = \frac{3}{2}, \frac{5}{2} $ and $ m_j = -\frac{5}{2}, -\frac{3}{2}, -\frac{1}{2} \frac{1}{2}, \frac{3}{2}, \frac{5}{2} $

    For $ j=\frac{3}{2} $, we have $ m_j = \frac{-3}{2}, \frac{-1}{2}, \frac{1}{2}, \frac{3}{2} $, i.e., 4 degenerate states.
    Also, the spectroscopic notation:
    \begin{equation*}
      3^2D_{3 / 2}
    \end{equation*}
    For $ j=\frac{5}{2} $, we have $ m_j = \frac{-5}{2}, \frac{-3}{2}, \frac{-1}{2}, \frac{1}{2}, \frac{3}{2}, \frac{5}{2} $, i.e., 6 degenerate states.
    Also, the spectroscopic notation:
    \begin{equation*}
      3^2D_{5 / 2}
    \end{equation*}
  \item 
    It is impossible to have this state because the maximum possible value for $ \ell $ is $ \ell = n-1 $, and since $ n=2 $ in this state, it is impossible for $ \ell = 2 $, i.e., the $ D $ subshell.
  \item 
    The ground state of the $ H $ atom is not split into two sub-levels by spin-orbit coupling because $ \ell=0 $, and there is only one electron, thus there is no internal magnetic field present to interact with the electron's magnetic moment.
  \end{enumerate}
\item 
  \begin{enumerate}
  \item 
    \begin{equation*}
      a_0 = \frac{4\pi \epsilon_0 \hslash^2}{m_e e ^2}
    \end{equation*}
    \begin{align*}
      E_I &= \frac{e^2}{4 \pi \epsilon_0} \frac{1}{2 a_0} \\
      &=  \frac{e^2}{4 \pi \epsilon_0} \frac{m_e ^2 e^2}{8\pi \epsilon_0 \hslash^2} \\
      &=  \frac{1}{2} \frac{e^4}{16 \pi^2 \epsilon_0^2 \hslash^2 c^2} m_e ^2 c^2 \\
      &=  \frac{1}{2} \left(\frac{e^2}{4 \pi \epsilon_0 \hslash c}\right) ^2 m_e ^2 c^2 \\
      &=  \frac{1}{2} \alpha^2 m_e ^2 c^2
    \end{align*}
  \item 
    We have the relativistic energy:
    \[
      E = \sqrt{\left( \mu c ^2 \right)^2 + \left( pc \right)^2}
    \]
    And we know that the rest energy, $ E_0 $:
   \[
     E_0 = \mu c ^2
   \]
   Thus we can find Kinetic energy by subtracting the rest energy from the total relativistic energy.
   \begin{align*}
     \text{KE} &= \sqrt{\left( \mu c ^2 \right)^2 + \left( pc \right)^2} - \mu c ^2 \\
        &= \sqrt{ \mu^2 c ^4  +  \frac{p^2c ^2 \mu^2 c^4}{\mu^2 c^4}} - \mu c ^2 \\
        &= \mu c ^2\sqrt{ 1  +  \left(\frac{pc}{\mu c^2}\right)^2} - \mu c ^2 \\
        &\approx \mu c ^2\left[ 1  +  \frac{1}{2}\left(\frac{pc}{\mu c^2} \right) ^2 - \frac{1}{8}\left(\frac{pc}{\mu c^2} \right)^4\right] - \mu c ^2 && \text{(using a Taylor series expansion)} \\
        &= \mu c ^2\left[ \frac{1}{2}\left(\frac{pc}{\mu c^2} \right) ^2 - \frac{1}{8}\left(\frac{pc}{\mu c^2} \right)^4\right] \\
        &= \frac{p^2}{2 \mu} - \frac{1}{8} \frac{p ^4}{\mu^3c^2} \\
        &= - \frac{p ^4}{8 \mu ^3 c ^2} && \text{(subtracting classical $ KE = p^2 / 2\mu $)}
   \end{align*}
   Thus, we have that the lowest order relativistic correction to the classical kinetic energy is 
   \[
     \Delta \text{KE} = \frac{p ^4}{8 \mu ^3 c ^2}.
   \]
  \item 
    Since $ j = \ell + s $, there are two possibilities for $ j $:
    \[
      j=\ell - \frac{1}{2} \quad \text{and} \quad j = \ell + \frac{1}{2}
    \]
    Thus,
    \[
      \ell=j + \frac{1}{2} \quad \text{and} \quad \ell = j - \frac{1}{2}
    \]
    Start with the $ \ell = j - 1/2 $ case:
    \begin{align*}
      f (j, \ell) &= \frac{j \left( j+1 \right) - 3 \ell \left( \ell + 1 \right) - \frac{3}{4}}{\ell \left(  \ell + \frac{1}{2} \right)\left( l+1 \right)} \\
      &= \frac{j \left( j+1 \right) - 3 \left( j - \frac{1}{2} \right) \left( j + \frac{1}{2} \right) - \frac{3}{4}}{\left( j - \frac{1}{2} \right) \left(  j \right)\left( j+\frac{1}{2} \right)} \\
      &= \frac{j^2 + j - 3 j^2 }{\left( j - \frac{1}{2} \right) \left(  j \right)\left( j+\frac{1}{2} \right)} \\
      &= \frac{j \left( 1-2j \right)}{\left( j - \frac{1}{2} \right) \left(  j \right)\left( j+\frac{1}{2} \right)} \\
      &= \frac{ 1-2j }{\left( j - \frac{1}{2} \right) \left( j+\frac{1}{2} \right)} \\
      &= \frac{ -2 \left( j - \frac{1}{2} \right) }{\left( j - \frac{1}{2} \right) \left( j+\frac{1}{2} \right)} \\
      &= - \frac{ 2  }{ \left( j+\frac{1}{2} \right)} \\
    \end{align*}
    Now, start with the $ \ell = j + 1/2 $ case:
    \begin{align*}
      f (j, \ell) &= \frac{j \left( j+1 \right) - 3 \ell \left( \ell + 1 \right) - \frac{3}{4}}{\ell \left(  \ell + \frac{1}{2} \right)\left( l+1 \right)} \\
      &= \frac{j \left( j+1 \right) - 3 \left( j + \frac{1}{2} \right) \left( j + \frac{3}{2} \right) - \frac{3}{4}}{\left( j + \frac{1}{2} \right) \left(  j + 1 \right)\left( j+\frac{3}{2} \right)} \\
      &= \frac{j \left( j+1 \right) - 3  j^2 - 6 j - 3 }{\left( j + \frac{1}{2} \right) \left(  j + 1 \right)\left( j+\frac{3}{2} \right)} \\
      &= \frac{-2 j ^2 - 5 j - 3}{\left( j + \frac{1}{2} \right) \left(  j + 1 \right)\left( j+\frac{3}{2} \right)} \\
      &= \frac{-2 \left( j^2 + \frac{5}{2} j + \frac{3}{2} \right)}{\left( j + \frac{1}{2} \right) \left(  j^2 + \frac{5}{2} j + \frac{3}{2} \right)} \\
      &= -\frac{2}{j + \frac{1}{2}}
    \end{align*}
    Hence, we have seen for both case that $ f(j, \ell) $ is actually independent of $ \ell $ and equal to 
    \[
      f (n, \ell) = - \frac{2}{j + \frac{1}{2}}.
    \]
  \end{enumerate}
\end{enumerate}

\end{document}

